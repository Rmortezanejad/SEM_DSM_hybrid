\chapter{Running through a Scheduler}\label{cha:Scheduler}

The code is usually run on large parallel machines, often PC clusters,
most of which use schedulers, i.e., queuing or batch management systems
to manage the running of jobs from a large number of users. The following
considerations need to be taken into account when running on a system
that uses a scheduler:
\begin{itemize}
\item The processors/nodes to be used for each run are assigned dynamically
by the scheduler, based on availability. Therefore, in order for the
\texttt{xgenerate\_databases} and the \texttt{xspecfem3D} executables
(or between successive runs of the solver) to have access to the same
database files (if they are stored on hard drives local to the nodes
on which the code is run), they must be launched in sequence as a
single job.
\item On some systems, the nodes to which running jobs are assigned are
not configured for compilation. It may therefore be necessary to pre-compile
both the \texttt{xgenerate\_databases} and the \texttt{xspecfem3D}
executables.
\item One feature of schedulers/queuing systems is that they allow submission
of multiple jobs in a ``launch and forget'' mode. In order to take
advantage of this property, care needs to be taken that output and
intermediate files from separate jobs do not overwrite each other,
or otherwise interfere with other running jobs.
\end{itemize}
Examples of job scripts can be found in the \texttt{\small utils/Cluster/}{\small{}
directory and can straightforwardly be modified and adapted to meet
more specific running needs.}{\small \par}

